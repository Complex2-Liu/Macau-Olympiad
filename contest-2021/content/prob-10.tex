\begin{prob}
\label{prob:prob-10}
\addcontentsline{toc}{section}{10. Problem 10}
已知在 $\triangle ABC, \angle A < \angle B < 90^\circ$.
圆 $\Gamma$ 过点 $A,B,C$. 圆 $\Gamma$ 过点 $A,C$ 的两条切线交于 $P$.
直线 $AB$ 与 $PC$ 交于 $Q$.
若三角形 $ACP, ABC, BQC$ 有相同的面积,
求证 $\angle BCA = 90^\circ$.
\end{prob}

\begin{center}
\includegraphics{prob-10.pdf}
\end{center}

\begin{soln}
我们用 $[XYZ]$ 表示 $\triangle XYZ$ 的面积. 我们有
\begin{equation}
\label{eq13}
\begin{aligned}
\angle PCA&=\angle PAC=\angle B,\\
\angle CPA&=180^\circ - 2\angle B,\\
\angle CQB&=180^\circ - \angle CPA-\angle BAP = \angle B - \angle A,\\
\angle QCB&=\angle A.
\end{aligned}
\end{equation}
因为 $[CQB] = [ACB]$, 且它们的高相等,
所以 $QB = c$. 因为 $QC$ 是切线,
所以 $QC^2 = QB\cdot QA \implies QC =\sqrt{2}c$.
由 $\triangle QCB \sim \triangle QAC$ 我们还可得出 $b = \sqrt{2}a$.
注意到
\[
\frac{1}{2}BA\cdot BC\cdot \sin\angle B = [ACB]
= [APC] = \frac{1}{2}CA\cdot CP\cdot \sin \angle B,
\]
this implies that
\[
PA = PC = \frac{ac}{b} = \frac{c}{\sqrt{2}}.
\]
另一方面, 我们还有
\[
\cos\angle B = \frac{\frac{1}{2}AC}{CP} =
\frac{b}{\sqrt{2}c} = \frac{a}{c}.
\]
这就证明了 $\angle BCA = 90^\circ$.
\end{soln}

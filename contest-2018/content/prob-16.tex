\begin{prob}
\label{prob:prob-16}
\addcontentsline{toc}{section}{16. Problem 16}
在锐角三角形 $\triangle ABC (AC \ne BC)$ 中,
已知 $AE \perp BC$ 于点 $E, BF \perp AC$ 于点 $F$.
过点 $E, F$ 的圆 $\Gamma$ 且与边 $AB$ 切于点 $D$.
若 $[ADE] = [BDF]$, 求证: $\angle EDF = \angle ACB$.
\end{prob}

\begin{soln}
设 $H$ 为 $\triangle ABC$ 的垂心.
设 $M$ 为 $AB$ 的中点. 设 $N$ 为 $CH$ 的中点.
设 $O$ 是 $\triangle ABC$ 的外心, $O'$ 是 $\Gamma$ 的圆心.

\begin{center}
\includegraphics{prob-16.pdf}
\end{center}

我们先分析条件 $[ADE] = [BDF]$ 告诉了我们什么信息:
\[
\frac{1}{2}AD\cdot AE \cos B = [ADE]
= [BDF] = \frac{1}{2} BF\cdot BD \cos A.
\]
利用 $BF = AB \sin A, AE = AB\sin B$ 可以推出
\begin{equation}
\label{eq:6}
\frac{DA}{DB} = \frac{\sin(\pi - 2A)}{\sin(\pi - 2B)}.
\end{equation}
另一方面, 观察 $\triangle AOB$, 我们有
\begin{equation}
\label{eq:7}
\frac{DA}{DB} = \frac{\sin \angle AOD}{\sin \angle BOD}.
\end{equation}
注意到 $\angle AOD + \angle BOD = 2 \angle C =
(\pi - 2\angle A) + (\pi - 2\angle B)$, 结合 \cref{eq:6} and \cref{eq:7}
便推出 $\angle AOD = \pi - 2A, \angle BOD = \pi - 2B
\implies \angle MOD = \angle B - \angle A$.

\bigskip

根据 Newton 定理 $M, N, EF$ 的中点三点共线, 由此知
$M, O', N$ 三点共线 (recall that $N$ is the circumcenter of $C, F, H, E$).
It is well-known that $OMNC$ is a parallelogram, hence
$\angle OMO' = \angle OMN = \angle OCN = \angle B - \angle A = \angle MOD$.
这意味着 $OMDO'$ 是一个矩形 $\implies O'D = OM = CN \implies O'$
是 $N$ 关于 $EF$ 的 reflection.
故 $2 \angle ACB = \angle ENF = \angle EO'F = 2 \angle EDF$.
\end{soln}

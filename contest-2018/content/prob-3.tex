\begin{prob}
\label{prob:prob-3}
\addcontentsline{toc}{section}{3. Problem 3}
试求方程 $x(x^2 + x + 1) = -\frac{1}{3}$ 的所有实数解 $x$.
\end{prob}

居然考三次方程的求根, 有点意思.
这里我们介绍一下 general 的三次方程求根理论,
下面的内容选自 \cite{cubic-equation}.

\begin{defin}[depressed cubic form]
\label{defin:depressed-cubic-form}
Cubics of the form
\[
t^3 + pt + q
\]
are said to be \textbf{depressed}.
\end{defin}

\begin{lem}
\label{lem:change-to-depressed-form}
Let $F = ax^3 + bx^2 + cx + d$ be an arbitrary cubic form,
then we can perform a change of variable after which $F$
become depressed. Moreover, this change of variable is linear in $x$.
\end{lem}

\begin{proof}
$x = t - \frac{b}{3a}$ works.
\end{proof}

\begin{defin}[Discriminant]
\label{defin:disc}
设 $f(x) = x^n + a_{n-1}x^{n-1} + \cdots + a_0$ 为一个\emph{首一}的多项式.
在 $\CC$ 中 $f(x)$ 分解为一次因式的乘积\footnote{代数基本定理, 这个事实会在复变函数以及拓扑课程中给出证明}
\[
f(x) = (x - r_1) \cdots (x - r_n).
\]
Then the \textbf{discriminant} of $f(x)$ is defined to be
\[
\mathrm{disc}(f) := \prod_{i < j}(r_i - r_j)^2.
\]

\begin{lem}
\label{lem:depressed-form-disc}
The discriminant of depressed cubic form $t^3 + pt + q$ is $-(4p^3 + 27q^2)$.
\end{lem}

\begin{proof}
根据 Vieta's formulas $r_1r_2 + r_2r_3 + r_3r_1 = p, r_1r_2r_3 = -q$.
直接验证
\[
(r_1 - r_2)^2(r_2 - r_3)^2(r_3 - r_1)^2 =
-4(r_1r_2 + r_2r_3 + r_3r_1)^3  - 27(r_1r_2r_3)^2.
\]
\end{proof}

\begin{prop}
\label{prop:nature-of-roots}
如果 $f(t) = t^3 + pt + q$ 是实系数多项式,
并且它的 discriminant $\mathrm{disc}(f) \ne 0$
(这相当于说 $f(t)$ 无重根), 则
\begin{enumerate}[label={\normalfont(\alph*)}]
\item $\mathrm{disc}(f) > 0 \implies f(t)$ 有三个不同的实根.
\item $\mathrm{disc}(f) < 0 \implies f(t)$ 有一个实根和两个互相共轭的复根.
\end{enumerate}
\end{prop}

\begin{proof}
注意复根必然成对出现. 若 $r_2, r_3$ 是共轭的复根,
则 $r_2 - r_3$ 是纯虚数, $r_1 - r_2, r_1 - r_3$ 是一对共轭的复数,
于是 $(r_2 - r_3)^2$ 是负实数, $(r_1 - r_2)(r_1 - r_3)$ 是正实数
(正是 $r_1 - r_2$ 这个复数的 norm 的平方).
所以此时 $\mathrm{disc}(f) < 0$.
\end{proof}

\begin{prop}[Vieta's substitution]
\label{prop:vieta-substitution}
The substitution $t = y - \frac{p}{3y}$ transforms the depressed cubic into
\[
y^3 + q - \frac{p^3}{27y^3} = 0.
\]
Multiplying by $y^3$, one gets a quadratic equation in $z := y^3$
\[
z^2 + qz - \frac{p^3}{27} = 0.
\]
\end{prop}

\begin{soln}
原方程为 $x^3 + x^2 + x + \frac{1}{3} = 0$,
代入 $x = t - \frac{1}{3}$ 得到 depressed form
$t^3 + \frac{2}{3}t + \frac{2}{27} = 0$.
再使用 Vieta's substitution $t = y - \frac{2}{9y}$ 得到
$y^3 + \frac{2}{27} - \frac{8}{729y^3} = 0$.
最后再令 $z = y^3$, 我们有 $z^2 + \frac{2}{27}z - \frac{8}{729} = 0$,
解得 $z = \frac{2}{27}$ or $-\frac{4}{27}$, 还原回去便有
\[
t  = \frac{2^{\frac{1}{3}} - 2^{\frac{2}{3}}}{3} \implies
x =  \boxed{\frac{1}{3}(-1 + 2^{\frac{1}{3}} - 2^{\frac{2}{3}}).}
\]
事实上, 从 depressed form 的 discriminant
我们也可以知道只有一个实根 (c.f. \cref{prop:nature-of-roots}).
\end{soln}
\end{defin}

\begin{prob}
\label{prob:prob-3}
\addcontentsline{toc}{section}{3. Problem 3}
设 $a,b,c$ 为正数且满足 $a + b + c = 2$. 求证:
\[
\frac{(a-1)^2}{b} + \frac{(b-1)^2}{c} + \frac{(c-1)^2}{a} \ge
\frac{1}{4}\left(\frac{a^2+b^2}{a+b} + \frac{b^2+c^2}{b+c} + \frac{c^2+a^2}{c+a}\right).
\]
\end{prob}

\begin{soln}
首先注意到右边是对称的, 但左边只是轮换的.
我们希望把轮换的变成对称的. 利用 $a + b + c = 2$ 我们有
\begin{align*}
\text{LHS} &= \cycsum \frac{a^2 - 2a +1}{b}
            = \cycsum \frac{a^2 - (a+b+c)a + 1}{b}\\
           &= \cycsum \frac{-ab-ac+1}{b}
            = \cycsum \left(-a - \frac{ac}{b} + \frac{1}{b}\right)\\
           &= -\cycsum a - \cycsum \frac{ac}{b} + \cycsum \frac{(a+b+c)^2}{4b}\\
           &=\frac{1}{4}\cycsum\frac{a^2+c^2}{b} +
             \frac{1}{4}\cycsum b - \frac{1}{4}\cycsum \frac{2ac}{b}\\
           &= \frac{1}{4}\cycsum \frac{(a-c)^2}{b} + \frac{1}{4}\cycsum b.
\end{align*}
其次我们还有
\begin{align*}
\cycsum \left(b - \frac{a^2+b^2}{a+b}\right)
&= \cycsum \left(b - (a+b) + \frac{2ab}{a+b}\right)\\
&= \cycsum \left(-\frac{a+b}{2}+\frac{2ab}{a+b}\right)\\
&= \cycsum -\frac{1}{2(a+b)}(a-b)^2.
\end{align*}
于是, 问题归结于证明
\begin{equation}
\label{eq:sos}
\cycsum \left(\frac{1}{a} - \frac{1}{2(b+c)}\right)(b-c)^2
=: \cycsum S_a(b-c)^2 \ge 0.
\end{equation}
这正是 S.O.S 标准形式.

\begin{thm}[S.O.S]
\label{thm:sos}
Consider the expression
\[
S = f(a,b,c) = S_a(b-c)^2 + S_b(c-a)^2 + S_c(a-b)^2,
\]
where $S_a, S_b, S_c$ are functions of $a,b,c$.
Then $S \ge 0$ if any of the following take place:
\begin{enumerate}[label={\normalfont(\roman*)}]
\item $S_a, S_b, S_c \ge 0$.
\item $a \ge b \ge c$ and $S_b, S_b + S_c, S_b + S_a \ge 0$.
\item $a \ge b \ge c$ and $S_a, S_c, S_a + 2S_b, S_c + 2S_b$.
\item $a \ge b \ge c$ and $S_b, S_c \ge 0, a^2S_b + b^2S_a \ge 0$.
\item $S_a + S_b + S_c \ge 0$ and $S_aS_b + S_bS_c + S_cS_a \ge 0$.
\end{enumerate}
\end{thm}

\begin{subproof}
See \cite{sosmethod}.
\end{subproof}

因为 \cref{eq:sos} 是对称的, 我们可以不妨假设 $a \ge b \ge c > 0$.
显然 $S_b, S_c \ge 0$, 最后再注意到
\[
S_b + S_a = \left(\frac{1}{a} - \frac{1}{2(c+a)}\right) +
\left(\frac{1}{b} - \frac{1}{2(b+c)}\right) \ge 0.
\]
Then we are done by part (ii) of \cref{thm:sos}.
\end{soln}

\begin{rem*}
这道题花了我最久的时间, 可能是因为我不等式太菜了.
一开始觉得不会很难估计能轻松凑个局部出来, 结果根本凑不动,
各种代数变形后阴差阳错得到了一个 S.O.S 形式.
\end{rem*}

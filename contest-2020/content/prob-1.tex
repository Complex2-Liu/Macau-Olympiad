\begin{prob}
\label{prob:prob-1}
\addcontentsline{toc}{section}{1. Problem 1}
5 位朋友各有一枚金币, 他们同时把自己唯一的一枚金币送予其他四位朋友之一,
当完成交换金币后这称之为一种操作.
问: 有多少种不同的操作使得操作后这 5 位朋友各仍有一枚金币?
\end{prob}

\begin{soln}
一种操作相当于一个映射
$\varphi: \{1,2,3,4,5\} \to \{1,2,3,4,5\}$,
第 $i$ 位朋友将自己的金币送予第 $\varphi(i)$ 位朋友.
条件仍各有一枚金币即 $\varphi$ 是一个排列 (双射),
条件送予其他四位朋友之一即 $\varphi(i) \ne i$ (没有不动点).
问题归结为错排公式 (c.f. \cite{derangement})
\[
D_n = n!\sum_{i=0}^n \frac{(-1)^i}{i!}.
\]
令 $n = 5$, 答案为 $\boxed{44.}$
\end{soln}

\begin{rem*}
注意到 $4! = 24, 5! = 120$, 所以刚好不能穷举.
\end{rem*}
